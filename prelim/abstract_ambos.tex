\documentclass[../main.tex]{subfiles}
% Resumen del documento
\begin{document}
\justifying
\section*{RESUMEN}
\begin{nohyphens}
\noindent
Este artículo contribuye a la literatura sobre los determinantes de la migración, centrándose en la relevancia de indicadores nacionales sobre los Volúmenes Internacionales de Migrantes. En contraste con la literatura existente, hago un análisis general, considerando a todos los países del mundo como posibles destinos de los migrantes. Implemento un modelo econométrico sobre datos de corte transversal del stock migratorio en 2015 para los territorios incluidos en las bases de datos del Banco Mundial y estimo modelos lineales mediante MCO. Encuentro que un ingreso per cápita más alto, menos tiempo requerido para iniciar un negocio y mayor control de la corrupción son factores positivos en el stock de migrantes. Un país más culturalmente diverso se relaciona con niveles más altos de migrantes, posiblemente porque facilita su integración social y económica. La democracia es significativa para los modelos, sin embargo, tiene un coeficiente negativo, lo cual es contradictorio con la literatura sobre migración. Con formas funcionales más complejas, descubro que los países ubicados en Medio Oriente y África del Norte, donde también se concentran la mayoría de países ricos y exportadores de petróleo, tienen flujos migratorios más altos pero puntajes de democracia bajos. Al analizar la democracia en el hemisferio occidental, se encuentra que el efecto es positivo sobre la migración, pero solo para países con un ingreso per cápita sobre aproximadamente 28 mil dólares PPP de 2017. La democracia posiblemente es una variable endógena e insta a mayor investigación para su estimación insesgada.
\end{nohyphens}
\noindent \textbf{Palabras clave}: Migración, PIB per cápita, Diversidad étnica, Corrupción, Democracia, Medio Oriente y África del Norte, Hemisferio occidental, Modelos lineales.
\clearpage
\section*{ABSTRACT}
\noindent
This paper contributes to the literature on the determinants of migration, focusing on the relevance of national level indicators on the International Migrant Stock. In contrast to the existing literature, I make a general analysis, considering all the countries around the world as possible destinations for migrants. I implement an econometric model over cross-sectional data for 2015 on territories included in the World Bank databases and estimate linear models through OLS. I find that a higher income per capita, less time required to start a business and more control of corruption are positive factors on the international migrant stock. A country more culturally diverse is related to higher migrant levels, perhaps because it facilitates social and economic integration. Democracy is significant for the models; however, its coefficient is negative, which is contradictory with literature on migration. With more complex functional forms, I discover that countries located in Middle East and North Africa, where most rich-oil exporting countries are located, have higher migration flows yet lower democracy scores. When analyzing democracy in the western hemisphere, it is found that the effect is positive on migration, but just for countries with an income per capita approximately over 28 thousand 2017 PPP dollars. Democracy is likely endogenous and calls for further investigation for its unbiased estimation.
     
\noindent \textbf{Keywords}: Migration, GDP per capita, Ethnic diversity, Corruption, Democracy, Middle East and North Africa, Western hemisphere, Linear models.
\end{document}