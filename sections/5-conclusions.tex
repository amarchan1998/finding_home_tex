\documentclass[../main.tex]{subfiles}
% Conclusiones
\begin{document}
\section{CONCLUSIONS}
The empirical models featured in this work show that, overall, income per capita is a very important determinant for migrant stocks around the world. A 1\% increase in income per capita is mostly related to percentage point increases in migrant stocks, except when considering only the western hemisphere of the world. Here, income only causes increases in migrants with more democratic countries, and with autocratic regimes it actually causes decreases in migration. This signals the importance democracy has in determining migration flows around the world. 

Economic freedom is also important and consistently positive for migrant stocks. A country that imposes less regulation to its new businesses usually also has high migrant stocks. This could be due to an easier introduction into labor markets for migrants, as businesses have relaxed recruitment procedures. I follow the literature and allow the perceptions of corruption control to affect democracy, and find also that it is a consistently positive factor for migration, meaning that countries with ‘cleaner’ governments have high migrant inflow. It is important to consider that corruption is closely related with economic freedom, according to the literature. Both should be kept together in all models to ensure the zero conditional mean assumption for the corruption coefficient estimation.

I also find that there is very high migration in the Middle East and North Africa (MENA), which is difficult to explain with other covariates. This might be due to the critical humanitarian situations in the region, which cause heavy migration between neighboring countries. A dummy variable for rich oil exporting countries is significant too as a replacement for this region dummy. This suggests that the mechanics of migration in oil exporting countries work differently than those in the rest of the world. Income still proves to be a positive effect on migrant stocks, as inside MENA and for rich oil exporters, income is associated with a greater increase of migrant stocks, relative to countries outside these groups.

I implemented culture in the estimation process by using the Historical Index of Ethnic Fractionalization, and it proved to be significant in most models with a positive sign. This also supports the determinations made by other researchers regarding migration determinants: a more diverse country is attractive to migrants since it will ease their economic and social transition in the foreign country. It is not significant, however, when I consider a model using a reduced sample with the western hemisphere and with a model considering the rich oil exporters dummy, probably due to multicollinearity or low variance.

Immigration policy variables as well as government expenditure did not keep their significance in the models, probably since other political variables as democracy may contain the information they include. The policy dummies, however, suffer from a lack of robustness in the sample, as not enough governments publicly announce their motivation to affect migrant levels.

The political regime score, or degree of democracy, as reported by \textcite{OurWorldInData.2015}, constantly proves significance, yet yields a counterintuitive sign. Supposedly, a more democratic country sees reductions in its migrant stock. This effect is reduced when partialling out the high average migration in special groups of countries: MENA and rich oil exporters. However, the partial effect of democracy is still negative and significant, either jointly or individually, when specifying it through variable interactions. Interacting democracy with special group dummies and income per capita uncovers the apparent inconsistencies with the sign, showing that, other things equal, with higher income, more democratic countries face reductions in migrant stocks. Further, more democratic rich oil exporters see reductions in their migrant stocks. Removing MENA countries does not fully eliminate the negative sign on the coefficient, yet it does reduce its magnitude. It is likely that this is due to the fact that democratic regimes inside MENA are also regimes that face greater political instability and violence, suggesting that migrants there might look for stability in autocratic regimes. However, it is not clear if this is an unbiased estimation of the effect of democracy on migrant stocks. Considering the western hemisphere on its own, democracy is positively related with migrant stocks only for relatively high income countries, signalling that democracy depends on the relative richness of a country for it to be a positive effect on migration. Migrants in the western hemisphere may try to ‘balance’ democracy and income per capita, and they seem to prefer higher values of both. 

The conclusions drawn here regarding democracy are fragile, as all points to this variable being endogenous. Multiple variables correlated with democracy may still be inside the error term, as they are unobservable. A true policy motivation of regimes is important, as it might be that democratic regimes are more likely to have been attractive destinations in the past, thus in the present have adopted restrictive immigration policies. Nevertheless, they do not reveal political purposes, as democratic regimes can be subject to more criticism for controversial positions than autocratic regimes. Immigration dummies also seem to fail to account for time trends on policy. Distance is also a variable that cannot satisfactorily be included in an empirical strategy of this kind; thus, these effects are left in the error term. However, switching to another strategy to use distance does have a cost, as distance measures require strong statistical capabilities of governments, which is not true for many underdeveloped nations, which are the ones that produce the most emigration.

To overcome this limitation, the use of proxies for the omitted variables may be satisfactory, however, the availability of data must also be considered for producing research. Many underdeveloped countries do not account with these variables and thus models are not representative enough to produce externally valid results. An instrument for democracy may also be used to cover the possibility of simultaneity between democracy and migrant stocks. Alternatively, an analysis with different dependent variables might uncover different relationships with democracy. This could show that perhaps only certain types of migration are affected negatively by democracy whereas other kinds of migration are not. Trying to separate migration based on the age of migrants could be very useful, as literature consistently points to its significance; however, once again availability of data becomes an issue. It must also be considered that the international migrant stock has potential to be an intensively short-term variable, whereas other ones, as the share of working age migrants can measure a long-term migration, i.e. migrants that left their country a long time before the statistic was reported. These kinds of migration may work through different mechanisms. In order to account for time trends on migration, and perhaps take advantage of exogenous shocks to it, as the COVID-19 pandemic most likely has caused, a panel-data approach can be valuable.

Ultimately, I identify important opportunities for progress in the literature but also interesting relationships between migration and economic statistics, one being the possibility of a desire of  ‘equilibrium’ between income per capita and democracy. The importance of the determinants of migration cannot be ignored, not only for countries already taking in high levels of migrants, but also for policymaking on countries that desire to stop the levels of emigration.  In order to stop citizens from leaving countries, governments should ideally make policy to help the country resemble nations that are net migrant intakers. The perceptions of the political processes are also important for both migration and economic growth: certainly reducing corruption helps the efficiency and institutionality of the public sector, but also gives the image of a stable country in which its citizens would rather remain. In the road to exhaustive economic wellness and the reduction of income disparity, migration proves to be an important factor, thus it is crucial to continue researching about this topic.

\end{document}