\documentclass[../main.tex]{subfiles}
% Revisión de Literatura
\begin{document}
\section{LITERATURE REVIEW}
According to \textcite{Amrith.2014}, migration occurs mainly for two general reasons: when people cannot satisfy their necessities where they live or when they are looking for new opportunities and growth. However, migration might also be driven by the needs of the family rather than the needs of the migrant. The empirical work on the determinants of migration has taken into consideration both economic and non-economic variables to get a better understanding of it and the policies that could be applied to foster sustainable migrant flows. 

\textcite{Mayda.2005} studied migration determinants for fourteen OECD countries between 1980 and 1995. She found that geography and demographic factors, such as distance and the share of young population at the origin, are the most important non-economic determinants for migration. Common language and past colonial relationships are not statistically significant factors. Countries with a bigger share of young population have a positive and significant impact on emigration rates. Further, it is suggested that positive pull factors, especially greater income opportunities, are bigger than average for destination countries when their policies of migration are less restrictive; push factors are negative and significant when migration policies relax. 

\textcite{Wesselbaum.2018} reached similar conclusions by analyzing the same group of countries over a larger time span. However, he considered proxies for education and health system, like years of schooling and life expectancy, and found that both are significant for the destination country. Surprisingly, it seems that tertiary level of education reduces incentives of immigration. The study added human capital to the analysis, and it is discovered that higher values of it can make countries less attractive for underskilled migrants. This might be since higher average human capital implies larger gaps between natives and migrants, consequently increasing the difficulty for them to find a job. 

When studying determinants for brain migration, by implementing a model that explains the rate of emigration of skilled workers in small states of world regions in 1990 and 2000, \textcite{Beine.2008} suggested that the violation of property rights is a significant influence for migration away from origin, as well as political instability, which is especially true for skilled migrants. In addition, religious fractionalization seems to be much more sensitive for small states, this means that, in small states, for a certain level of religion fractionalization, people are three more times willing to migrate, relative to bigger states. 

When focusing on the share of young population of migrants,\textcite{Wesselbaum.2018}  contributed with the idea that this share may be an important determinant of migrant flows, as younger people may be more willing to emigrate. At country of destination, he found that population density matters, because immigrants will avoid countries where labor competition is higher, as there they could face conflict over scarce resources. Additional to this, it was established that human capital has a U-shaped effect on destination countries. On the contrary, by analyzing migrant’s choice of destination inside Nepal, based on Census and Living Standards Surveys Data, \parencite{Fafchamps.2012} found that, in that region, people tend to emigrate to \enquote{high population density areas that are nearby, have good access to amenities, higher average income and consumption, higher housing premium, and where many people share their language and ethnic background} (p.15). Additionally, they mention that the principal reason for moving from one district to another for women is marriage, for children and youth’s education, and for adult men it is work (p.6).

All the literature mentioned before considered income as one of the most relevant determinants for migration. \textcite{Stark.1991} explored this factor by studying rural Mexican households in order to find out in what people base their migration decision. One alternative was that they based it on the possibility of higher income for their household. Also, they instead might have been motivated by relative deprivation, which means that they were interested in putting their household in a better position, compared with a specific reference group in their village or area. For international migration, \enquote{relatively deprived households are more likely to emigrate than households that are more favorably situated in their village’s income distribution} (p. 1176). For internal migration between rural and urban areas, both factors related to income have no direct effects over households’ decision to emigrate; this happens because of the perception that it is riskier and more expensive to migrate to a destination where a reference group substitution is possible.

Culture not only plays a fundamental role in economic, political, and social institutions of a country, but also in migration. When talking about international migration it is necessary to consider the role of this factor, since the differences between the culture at origin and at destination may affect migrant flow mechanisms. It would be expected that migrants choose destination countries with a culture similar to their own, or at least a country with a higher level of cultural diversity. The latter could be perceived as more likely to receive migrants and offer opportunities, encouraging harmonious cultural integration. An interesting and different approach about the importance of cultural integration for migration was made by \textcite{Cameron.2012}. They designed laboratory experiments to analyze how migrant’s preferences and behaviors change over time when they are living in a country different than their own. They conducted these experiments with Chinese participants living in Australia and suggested that \enquote{exposure to Western education has a significant impact on social preferences, preferences for competition, and risk attitudes} (p.24). Additionally, they find that the best approach for cultural integration is through education; however, for some countries like Australia it is more significant to promote multiculturalism rather than seeking a complete cultural integration. 

Some studies focus on capturing culture as a principal determinant for migration, as \textcite{Wang.2016}. Their main finding about culture is that the average cultural distance in a country is crucial for migrants; being three times more valuable than geographical proximity. The higher this factor is, the more attractive for younger migrants but less for older ones. Cultural distance is also taken into consideration by \textcite{Caragliu.2012}. In their work they find that \enquote{when cultural differences are based on the degree of post-materialist values, the effect of a greater distance seems to foster migratory movements} (p.20). While migrants will be more attracted to more culturally diverse countries, the distance of the home culture with the foreign is negatively related to the migration decision. This suggests that governments that desire to increase immigration inflows should smooth the transition of one culture to another, that is, to narrow the cultural distance between immigrants and the host society, and to make immigrants understand a country’s social norms, principles, and institutions. This does not mean that cultural diversity reduces the attractiveness of a country to immigrants, it is the nature of the cultural diversity that influences the decision. In addition, they include language as a possible determinant for migration and find that it is a positive factor for attracting migrants. Moreover, their results suggest that institutional and financial distance seem to present a negative effect on migration flows, as \textcite{Wang.2016} found. 

Democracy may also influence a migrant’s decision. Little research is concerned about this factor, however, \textcite{Azad.2020} established a relationship between democracy and migration which is relevant to my study. It is suggested that democracy has a \textit{positive} impact on migration, therefore, a country that is more democratic should be more attractive for migrants: \enquote{immigration rises by 29\% in the long-run due to democracy} (p.31). Likewise, \textcite{Prada.2020} determined that for refugees and vulnerable migrants, democracy seems to be positively correlated with migration for a single year cross section of the European Union. This suggests that for any origin or migration situation an immigrant is facing, besides of the economic well-being, it is essential for them to feel that their \enquote{rights and freedoms are respected} (p.477). In spite of her analysis, for future studies, she encouraged investigators to add corruption and economic freedom. An empirical approach considering national corruption perceptions was taken by \textcite{Dimant.2013}. Through their study, they demonstrated that high corruption drives skilled migration away and lowers the incentives for returning. Nevertheless, for average migration this finding is \enquote{less pronounced and not statistically robust} (p.1274).
\end{document}