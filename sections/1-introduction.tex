\documentclass[../main.tex]{subfiles}
% Introducción
\begin{document}
\section{INTRODUCTION}
Human migration is a global phenomenon that over the years has impacted different territories around the world, changing their ethnic, racial, and linguistic compositions \parencite{EncyclopaediaBritannica.2021b}. In general, immigrants not only look for an improvement in their quality of life and opportunities abroad, also they tend to look for ways to help their families at their origin country. It is important to know the mechanisms that determine migration, since according to \textcite{Murrugarra.2011}, migration has historically reduced poverty, especially at the country of origin, through the remittances sent from migrants abroad. Migration is often also featured in political discourse and the rationale for many policy decisions, as it can affect labor markets \parencite{Abel.2014} and thus affect economic behavior for all agents in them. 

To analyze global migration, the most intuitive way of looking at the migrant’s decision is
to consider relative richness between countries. It is naturally expected that migrants, above all,
contemplate richer countries as their prime destiny, since they are thought to be unsatisfied with the current economic or social conditions that them or their families endure in the home country. 
The most obvious way of doing this is by considering the relationship between income per capita and the number of immigrants in a country. According to \textcite{Sjaastad.1962}, one of first researchers to analyze the economic incentives of migration, the relationship between it and income is positive, yet in some cases, this relation is small due to the difficulties in analyzing migration’s effects over labor markets when considering net migration. Rather, \textcite{Sjaastad.1962} proposes a cost-benefit approach to study this connection, which is contingent on many factors of the individual migrant decision. In my work, I will analyze countries and not individuals, so the expected finding with basis to Sjaastad's work is a positive effect of income per capita over migrant stocks. However, it is important to consider other determinants based on culture, immigration policies, economic freedom, governance, and stability, as migration can be a direct consequence of social upheaval due to wars, natural disasters, violence, among others. 

This paper aims to explore the principal determinants of migration at a global level, using the international migrant stock from the World Bank World Development Indicators as the main variable to be investigated. Through an empirical approach, I aim to discover more about what drives migration in the planet, and how different social, political, and demographic circumstances affect migrant stocks. I estimate linear models through OLS that attempt to explain the main determinants for migration, focusing on a country’s attractiveness for migrants based on values of indicators. Unlike other studies, I analyze factors which are available at the national level, rather than focusing on specific regions or intranational migration. This means that most of conclusions drawn here, while not exactly separable for origin and destination countries, are relevant on the world level. Conclusions to this objective will be based on the statistical significance of the variables in the empirical models and, most importantly, on the sign of coefficients and how they fit into the research in this topic. 

My estimation results show that income per capita, economic freedom, ethnic diversity and the control of corruption are correlated with higher migrant stocks. The models, however, show a limitation when estimating the effect of democracy: while it would be expected to affect migration positively, the models yield the opposite relationship. This might be due to special relationships among variables in some regions of the world, especially the Middle East and North Africa. When restricting my analysis to countries in the western hemisphere I find that income per capita and democracy affect migration stocks jointly: richer and more democratic countries are correlated with higher migrant stocks. Richer countries only see positive partial effects for values of democracy over 4, that is, only for somewhat democratic regimes (anocracies). More democratic regimes, on the other hand, only see positive partial effects for countries with GDP per capita over 28 thousand 2017 PPP dollars, which is near the 58\textsuperscript{th} percentile of the world income distribution in 2015.

The following section includes a brief review of the literature on migration determinants. I later establish my empirical strategy and move on to discuss the estimation results. 

\end{document}